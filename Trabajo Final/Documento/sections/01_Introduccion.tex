\section{Introducción}

En un mundo donde la tecnología se ha entrelazado con casi todos los aspectos de nuestra vida diaria, la protección de datos y la privacidad se han convertido en preocupaciones primordiales. Desde nuestras interacciones en línea hasta nuestras transacciones financieras, cada acción deja un rastro digital que puede ser explotado si no se maneja adecuadamente.

Imagina por un momento que estás navegando en tu tienda en línea favorita y, sin darte cuenta, alguien está manipulando la información que el sitio web guarda sobre ti. Este proceso, conocido como \textbf{cookie poisoning} o \textbf{envenenamiento de cookies}, es una técnica sutil pero poderosa que los cibercriminales utilizan para robar datos, acceder a cuentas y causar estragos en el mundo digital.

El objetivo de este trabajo es desentrañar los secretos detrás del cookie poisoning. Exploraremos qué son las cookies, cómo pueden ser manipuladas para recopilar información personal sin nuestro conocimiento y consentimiento, y las implicaciones que esto tiene para nuestra privacidad en línea.

¿Te has preguntado alguna vez cómo un atacante puede acceder a tu cuenta sin conocer tu contraseña? ¿O cómo pueden robar tu información personal mientras navegas tranquilamente por internet? Pues sigue leyendo para descubrir cómo protegerte y comprender mejor una de las amenazas más insidiosas del mundo digital moderno.


