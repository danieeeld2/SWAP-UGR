\section{Antecedentes}

Este apartado revisará la literatura existente y presentará los conceptos fundamentales que sustentan el desarrollo de este trabajo, reforzando la necesidad de estudiar y comprender esta amenaza.

\bigskip

\textbf{Historia y Evolución de las Cookies}

Las cookies fueron introducidas por primera vez en 1994 por Lou Montulli, un programador de Netscape Communications, como una forma de gestionar la información de estado en el navegador web de los usuarios (\cite{kristol2001}). 

Originalmente diseñadas para mejorar la experiencia del usuario al recordar sus preferencias y facilitar el comercio electrónico, las cookies han evolucionado hasta convertirse en una herramienta crucial para la funcionalidad de la web moderna.

\bigskip

\textbf{Conceptos Fundamentales}
\begin{itemize}
    \item \textbf{Cookies}: Pequeños archivos de texto que los sitios web almacenan en el navegador del usuario para recordar información sobre la sesión del usuario, preferencias, y otros datos útiles \cite{rosenblum2001}.

    \item \textbf{Sesiones Web}: Según el estudio de Wang y Goldberg (2006) (\cite{wang2006}), es un mecanismo utilizado por las aplicaciones web para mantener el estado de un usuario entre diferentes solicitudes a un servidor. Las cookies de sesión son fundamentales para este proceso.

    \item \textbf{Protocolo HTTP}: El protocolo HTTP/1.1 se define en el RFC 2616 (Fielding et al., 1999) (\cite{fielding1999}). Se define como un protocolo de solicitud-respuesta utilizado en la comunicación entre el cliente y el servidor en la World Wide Web. 
    
    Funciona como un protocolo sin estado, lo que significa que cada solicitud se procesa de forma independiente, sin conocimiento de solicitudes anteriores. Las cookies se utilizan para mantener el estado de la sesión del usuario entre solicitudes, permitiendo a las aplicaciones web realizar seguimiento del estado de la sesión, recordar preferencias y mantener la autenticación del usuario. 
\end{itemize}

\bigskip

\textbf{Uso de cookies actualmente}

Actualmente los principales usos de las cookies son:

\begin{itemize}
    \item \textbf{Gestión de Sesiones} : Mantienen a los usuarios autenticados mientras navegan por un sitio web, recordando su estado de inicio de sesión.
    \item \textbf{Personalización} : Almacenan preferencias del usuario, como el idioma y temas visuales, mejorando la experiencia de navegación.
    \item \textbf{Análisis y Seguimiento} : Recopilan datos sobre el comportamiento del usuario para análisis de tráfico web y marketing dirigido.
    \item \textbf{Carros de Compra} : Guardan los artículos añadidos en los carros de compras en sitios de comercio electrónico, incluso si el usuario navega fuera del sitio.
\end{itemize}

Puesto que se almacenan datos sensibles es necesario una buena protección de las mismas y es uno de los temas que discutiremos posteriormente en este trabajo.





